\documentclass[a4paper, 11pt, margin=3cm]{article} % Font size (can be 10pt, 11pt or 12pt) and paper size (remove a4paper for US letter paper)

\usepackage[utf8]{inputenc}
\usepackage[swedish, english]{babel}

\makeatletter

\renewcommand{\maketitle}{ % Customize the title - do not edit title and author name here, see the TITLE block below
{
\raggedright
{\LARGE\@title} % Increase the font size of the title
}
\vspace{-45pt} % Some vertical space between the title and author name
\begin{flushright} % Right align
{\large\@author} % Author name
\\\@date % Date

\vspace{40pt} % Some vertical space between the author block and abstract
\end{flushright}
}

%----------------------------------------------------------------------------------------
%	TITLE
%----------------------------------------------------------------------------------------

\title{\textbf{Reflektion}\\ \vspace{5pt} % Title
Examensarbete} % Subtitle

\author{\textsc{Niklas Ericson}
\\ 19900527-1953 % Author
\\ LiTH-ISY-EX-{-}16/4994-{-}SE}

\date{\today} % Date

%----------------------------------------------------------------------------------------
%	DOCUMENT
%----------------------------------------------------------------------------------------

\begin{document}
\maketitle % Print the title section

\section*{Hur relaterar examensarbetet till programmålen}
Mitt examensarbete har varit väl relaterat till de mål som har satts upp för Y-programmet, Teknisk Fysik och Elektroteknik. Ett av målen är att Y-ingenjören, utifrån en stabil matematisk, naturvetenskaplig och teknikvetenskaplig grund ska kunna ta sig an och lösa komplexa tekniska problem vilket är vad det här exjobbet har handlat om, dvs. att applicera mina kundskaper på ett teknisk komplext problem. Exjobbet har också genomförts på ett systematisk sätt, använt matematiska och fysikaliska verktyg för att modellera och analysera system där kunskaper och metoder från Vågfysik, Mekanik, Reglerteknik mfl. har använts. Vilket väl överensstämmer med målen i utbildningsplanen.

I stort sett så har alla individuella och yrkesmässiga färdigheter utnyttjats i examensarbetet så som ingengörsmässigt tänkande, experimenterande, systemtänkande, ett självständingt, kreativt och kritisk förhållningssätt samt ett professionellt uppträdande. Det ingenjörmässiga tänkande har givetvis genomsyrat hela arbetet från början till slut, innefattande  identifiering av problem, relevanta antaganden och rimlighetsbedömningar. Där hypoteser har utvärderats i simuleringar och med relevanta experiment. Litteraturstudien i början av examensarbetet innehöll insamling av ny relevant kunskap som sedan applicerades på det kända problemet. Även detta är ett tydligt mål i utbildningsplanen.

Skriftlig såväl som muntlig kommunikation har under hela projektets gång uteslutande skett på engelska, också i linje med utbildningsmålen.

Examensarbetet har minst sagt gett perspektiv på teknikens betydelse i samhället. På CERN använder man modern teknik för att ta reda på universums uppkomst där uppfinningar alltid publiceras och oftast även kommercialiseras.

\section*{Det egna arbetet}

\section*{Ämnesinnehåll, kunskaper, färdigheter och förhållningssätt som var till mest nytta för examensarbetets genomförande}

\end{document}
