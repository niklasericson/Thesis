% !TEX root = main.tex
%en preliminär problemformulering satt i relation till litteraturbasen
\chapter{Introduction}\label{cha:intro}

\section{Background}
High precision positioning systems are vital in e.g. scanning tunneling microscopes (\abbrSTM), atomic force microscopes (\abbrAFM) and in semiconductor lithography. In \abbrAFM, for instance, high precision positioning is required to control the vertical position of the scanning probe to keep the force constant between the sample surface and the probe tip. A topographical image of the sample is obtained by raster-scanning the probe over the sample surface and plotting the vertical displacement against the probe's x-y position. A positioning system that keeps the force constant down to an atomic-scale resolution is thus inevitable in order to obtain a high resolution image without damaging the sample \citep{SurveyOfControlIssues:2007}.

The piezoelectric effect is a phenomenon that arises in certain solid materials when an electric potential is generated in response to applied mechanical stress. The effect was first discovered by Jacques and Pierre Curie in 1880 when they found that applying pressure to a quartz crystal generates electrical potential. Today, the effect is commonly encountered in daily life and utilized in for example lighters, buzzers and loudspeakers.

Smart materials such as piezoelectric and magnetostrictive materials are nowadays commonly used in precision actuators due to their ability to convert electrical energy into mechanical energy. Piezoelectric materials have been commercially available for almost 45 years and have become indispensable for the nanopositioning industry \citep{Piezo:2008}. In cases where a relatively small displacement range is required (travel ranges up to \unit{500}{\micro\meter}) a piezo electric device is the actuator of choice due to its fast response, high resolution and its ability to generate large mechanical forces for small amounts of power in compact designs \citep{SurveyOfControlIssues:2007}.

The \abbrENSTIECE section in the Engineering Department at \abbrCERN (European Organization for Nuclear Research) is developing a high precision positioning system for use in the UA9 crystal collimation study.

\section{Motivation}
Crystalline solids have the ability to constrain the directions that particles take as they pass through, this is commonly called the "channeling" property. The UA9 collaboration at \abbrCERN is investigating how tiny bent crystals can help to steer particle beams in modern hadron colliders such as the Large Hadron Collider (\abbrLHC) \citep{WebsiteUA9:2016}. In high energy colliders particles tends to drift outwards creating a beam halo. These particles surrounding the beam, might drift and cause damage to sensitive parts in the accelerator, such as the superconducting magnets which can suffer an abrupt loss in superconducting capability (quench), even from a small dose of deposited energy. To extract and absorb these halo particles, \abbrCERN uses a multi-stage collimation system, consisting of primary and secondary collimators connected in series. \abbrCERN's largest particle accelerator, the \abbrLHC operating at \unit{7}{\tera\electronvolt}, has 108 collimators distributed along 2 beam pipes \citep{CrystalCollimation:2015}. At the moment, these collimators use massive blocks of amorphous material to intercept with the beam and absorb halo particles. The UA9 experiment aims to develop a new collimator, utilizing the technique of a bent crystal and a single absorber which will, in theory, imply in a more efficient cleaning, a less complex system and a reduction of the machine impedance. These are all essential for reaching higher energy levels in a future particle accelerator.

\section{Purpose and Goal}
One major difficulty that arises with the use of bent crystals is that, the higher the energy of the particle, the lower the angular acceptance for channeling. Hence, a high precision rotational mechanism is required. For this purpose, the \abbrENSTIECE section is developing a rotational stage that will insert and rotate the crystal with a high angular accuracy. The purpose of this thesis is to identify possible control approaches that could be applicable to the rotational stage in order to achieve the desired performance. The stage is required to:
\begin{itemize}
  \item have a total range of \unit{20}{\milli\rad}.
  \item be able to track reference trajectories at ramp rates of \unit{100}{\micro\radianpersecond}.
  \item reject external disturbances to maintain a maximum tracking error of $\pm$\unit{1}{\micro\rad} even when the linear axis is moving.
\end{itemize}

\newpage
\section{Prospective Challenges}\label{sec:prospectiveChallanges}
First of all, piezoelectric actuators show strong nonlinear properties such as hysteresis and creep (drift), which have to be compensated for \citep{Piezo:2008}. Moreover, the mechanical flexural structure in combination with the piezoelectric characteristics leads to a highly resonant structure, making it difficult to achieve the desired performance while operating the rotational stage within noisy environments with external disturbances such as ground vibrations. Furthermore this rotational stage is attached to a linear stage which is composed by a lead-screw, a stepping motor and an axis. The linear movement adds additional perturbation to the rotational stage due to imperfections in the lead-screw and detent torque and stepping nature of the motor \citep{ButcherController:2015}. Finally, the system changes drastically due to a number of factors such as linear position, linear speed and angular position in combination with a moving center of rotation. The linear dynamic modeling is thereby limited by all these factors requiring a controller that is robust to such variations.

\section{Related Work}
During the last couple of decades, a lot of research has been put into the area of nanopositioning and its applications. A well known application is the \abbrAFM discussed in \citep{chuang2013robust, SurveyOfControlIssues:2007}, where piezoelectric actuators are commonly used to position the scanning probe tip. The piezoelectric actuator is in many aspects the actuator of choice but its nonlinear characteristics makes it hard to control. A lot of recently published reports discuss the control of piezoelectric actuator \citep{gu2013motion, gu2016modeling} and also how to model and compensate for the nonlinear behavior \citep{Biggio:2014,ButcherIdentification:2015,Maxwell:2012,leang2002hysteresis}.

For the purpose of increasing tracking capability, disturbance rejection and model error robustness in the area of nanopositioning a wide range of controllers have been reported with success. Many of these controller either aims directly to suppress the nonlinear behavior \citep{ompc, Elmali:1996, xu2014model} or works in combination with a hysteresis cancellation method \citep{inputshaper, gu:2014}.

For systems suffering from harmonic disturbances, control methodologies offering harmonic cancellation could be used to efficiently increase the overall tracking performance. Several methods exist where the harmonic cancellation is added either as a feedforward from the modeled disturbance \citep{fujimoto2009rro, vilanova2008disturbance} or directly in the closed loop \citep{IMP:Perry}.

At \abbrCERN, a first approach has already been implemented to achieve the desired performance. The proposed controller, presented in \citep{ButcherController:2015} delivers reasonable performance but does not fulfill the requirements during movement. The authors proposes a \abbrPID controller in combination with a prefilter, and a hysteresis compensator. The controller has shown high disturbance rejection at the first resonance peak as well as good tracking performance.

\section{Method}
This thesis aims to identify possible control approaches, benchmark them in simulations to confirm their applicability, and implement the most promising one. The goal can be stated in two questions which this thesis aims to provide answers for.

\begin{itemize}
  \item What control approaches can be used to achieve the desired performance?
  \item Which one is the most promising approach with respect to simulated/benchmarked results and ease of implementation on the real device?
\end{itemize}

To provide answers to the above questions the following methodology (presented in chronological order) has been used in this thesis.

\begin{itemize}
  \item Literature study.
  \item Further investigation of selected control approaches.
  \item Benchmarking tests of selected control approaches in simulations.
  \item Implementation of the most promising approach.
  \item Proposal of controller.
\end{itemize}

The literature study aimed to cover a large number of control approaches to give a good overview of identified advantages and disadvantages with each approach. After the initial literature study, further investigation was conducted with the selected approaches. All approaches were benchmarked in simulations, carried out in Matlab and Simulink. Finally, with performance, stability and implementation considerations, one approach was selected and implemented in LabVIEW to operate on the real device. The final results was then used to verify the simulation performance and to give a final statement of the controller's applicability and effectiveness.

\section{Limitations}
This thesis have solely focused on control approaches and have thereby excluded all modeling of the system. Extensive modeling had already been carried out on the device, motivating the exclusion. Due to time limitations, only a few control approaches was selected for simulation and only one of them was implemented on the real device. For the same reason, the controller tuning was only carried out until a sufficient set of parameters was found, even though a more optimal set of parameters could have existed at the time.

All simulations will be performed on a linear system model, assuming perfect inverse hysteresis cancellation and a sufficient closed loop to compensate for the creep effect, motivated by the extensive hysteresis model. Furthermore, all controllers will be discretized with a 2kHz sampling frequency for the sake of comparison, even though the number of operations might allow for a higher execution rate.

\section{Outline}
This thesis provides the reader with a detailed description of the system, related theory and simulation results of the selected control methodologies as well as implementation results from benchmarking tests with the selected harmonic cancellation approach. The overview presented in Chapter~\ref{cha:systemOverview} provides a brief introduction to collimation and how collimators operate in the \abbrLHC, a more detailed description of the rotational stage including the nonlinear effects that origin from the piezoelectric material. Moreover, a description of the linear identification and the nonlinear compensation is presented in the same chapter as well as an explanation of the present control algorithm. In Chapter~\ref{cha:controlApproach} the simulation results of each controller is presented individually, benchmarked only to the present system. This is followed by two comparison tables gathering comparable results from the stand alone methods and the harmonics cancellation methods, respectively. In Chapter~\ref{cha:exp_result} the experimental results from implementation of the harmonic cancellation is presented and in Chapter~\ref{cha:conclusion} the work is concluded and final answers to the formulated questions are given.
