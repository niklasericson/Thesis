% !TEX root = main.tex
%en preliminär problemformulering satt i relation till litteraturbasen
\chapter{Introduction}\label{cha:intro}

\section{Background}
High precision positioning systems are vital in e.g. scanning tunneling microscopes (\abbrSTM), atomic force microscopes (\abbrAFM) and in semiconductor lithography. In \abbrAFM, for instance, high precision positioning is required to control the vertical position of the scanning probe to keep the force constant between the sample surface and the probe tip. An topographical image of the sample is obtained by raster-scanning the probe over the sample surface and plotting the vertical displacement against the probe's x-y position. A positioning system that keeps the force constant down to an atomic-scale resolution is thus inevitable in order to obtain a high resolution image without damaging the sample \citep{SurveyOfControlIssues:2007}.

The piezoelectric effect is a phenomenon that arises in certain solid materials when an electric potential is generated in response to applied mechanical stress. The effect was first discovered by Jacques and Pierre Curie in 1880 when they found that applying pressure to a quarz crystal generates electrical potential. Today, the effect is commonly encountered in daily life and utilized in for example lighters, buzzers and loudspeakers.

Smart materials such as piezoelectric and magnetostrictive materials are nowadays commonly used in precision actuators due to their ability to convert electrical energy into mechanical energy. Piezoelectric materials have been commercially available for almost 45 years and have become indispensable for the nanopositioning industry \citep{Piezo:2008}. In cases where a relatively small displacement range is required (travel ranges up to \unit{500}{\micro\meter}) a piezo electric device is the actuator of choice due to its fast response, high resolution and its ability to generate large mechanical forces for small amounts of power in compact designs \citep{SurveyOfControlIssues:2007}.

The \abbrECE (Equipment Controls and Electronics) section in the Engineering Department at \abbrCERN (European Organization for Nuclear Research) is developing a high precision positioning system for use in the UA9 crystal collimation study.

\section{Motivation}
 Crystalline solids have the ability to constrain the directions that particles take as they pass through, this is commonly called the "channelling" property. The UA9 collaboration at \abbrCERN is investigating how tiny bent crystals can help to steer particle beams in modern hadron colliders such as the Large Hadron Collider (\abbrLHC) \citep{WebsiteUA9:2016}. In high energy colliders particles tends to drift outwards creating a beam halo. These particles surrounding the beam, can be lost and cause damage to sensitive parts in the accelerator, such as the superconducting magnets which can suffer an abrubt loss in superconducting capability (quench), even from a small dose of deposited energy. To extract and absorb these halo particles, \abbrCERN uses a multi-stage collimation system, consisting of primary and secodary collimators connected in series. \abbrCERN's largest particle accelerator, the \abbrLHC operating at \unit{7}{\tera\electronvolt}, has 108 collimators distributed along 2 beam pipes \citep{CrystalCollimation:2015}. At the moment, these collimators use massive blocks of amorphous material to intercept and absorb halo particles. The UA9 experiment aims to develop a new collimator, utilizing the technique of a bent crystal and a single absorber which will, in theory, imply in a more efficient cleaning, a less complex system and a reduction of the machine impedance. These are all essential for reaching higher energy levels in a future particle accelerator.

\section{Purpose and goal}
One major difficulty that aries with the use of bent crystals is that, the higher the energy of the particle, the lower the angular acceptance for channeling. Hence, a high precision rotational mechanism is required. For this prupose, the \abbrECE section is devoloping a rotational stage that will rotate the crystal with a high angular accuracy. This purpose of this thesis is to identify possible control approaches that could be applicable to the rotational stage in order to achieve the desired performance. The stage is required to:
\begin{itemize}
  \item hava a total range of \unit{20}{\milli\rad}
  \item be able to track reference trajectories at ramp rates of \unit{100}{\micro\radianpersecond}
  \item reject external disturbances to maintain a maximum tracking error of $\pm$\unit{1}{\micro\rad}
\end{itemize}

\section{Prospective challenges}
First of all, piezoelectric  actuators show strong nonlinear properties such as hysteresis and creep (drift), which have to be compensated for. Moreover, the mechanical flexural structure in combination with the piezo electric characteristics leads to a highly resonant structure, making it difficult to achieve the desired performance while operating the rotational stage within noisy environments with external disturbances such as ground vibrations. Furthermore this rotational stage is attached to a linear stage which is composed by a leadscrew, a stepping motor and an axel. The linear movement adds additional perturbation to the rotational stage due to imperfections in the leadscrew and detent torque and stepping nature of the motor. Finally the system dynamics also show linear position dependence requiring a controller that is robust to such variations.

\section{Related work}
One attempt to achieve the desired performance has already been made. The proposed controller, presented in \citep{ButcherController:2015} delivers reasonable performance but does not fulfill the requirements during movement. The authors proposes a PID controller in combination with a pre-filter, and a hysteresis compensator. The controller has shown high disturbance rejection at the first resonance peak as well as good tracking performance.

\section{Approach}

\begin{itemize}
  \item What are the possible control approaches that can be used to acheive the desired performance?
  \item Which one is the most promising approach with respect to simulated/benchmarked results and ease of implementaion on the real device?
\end{itemize}

\section{Limitations}
This thesis will solely focus on the control approaches

\section{Outline}
This thesis plan presents an overview of the thesis, including method, literature base and expected results. The method and work flow of the thesis as well as a comprehensive literature review is given in Chapter~\ref{cha:method}. In Chapter~\ref{cha:result} the results that can be expected half way through the project is discussed while a brief summary of the thesis can be found in Chapter~\ref{cha:conclusion}.
