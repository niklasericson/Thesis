The Equipment Controls and Electronics section at \abbrCERN is developing a high precision piezo-actuated rotational stage for the UA9 crystal collimation project. Several control-related issues arising from the complexity and operational environment of the system make it difficult to design a controller that achieves the desired performance. This thesis investigates in different control approaches aiming to improve the tracking capability of the rotational stage. It shows that the \abbrIRC method could be used to efficiently control the rotational stage. Moreover it shows that a harmonic cancellation method could be used to increase the tracking accuracy by canceling known harmonic disturbances. This harmonic cancellation (the \abbrRFDC) was implemented in this thesis and proposed as an add-on to the present control algorithm.

%If your thesis is written in English, the primary abstract would go here while the Swedish abstract would be optional.
