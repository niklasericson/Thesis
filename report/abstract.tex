The Equipment Controls and Electronics section (\abbrENSTIECE) at \abbrCERN is developing a high precision piezo-actuated rotational stage for the UA9 crystal collimation project. This collaboration is investigating how tiny bent crystals can help to steer particle beams used in modern hadron colliders such as the Large Hadron Collider (\abbrLHC). Particles are deflected by following the crystal planar channels, "channeling" through the crystal. For high energy particles the angular acceptance for channeling is very low, demanding for a high angular precision mechanism, i.e. the rotational stage. Several control-related issues arising from the complexity and operational environment of the system make it difficult to design a controller that achieves the desired performance. This thesis investigates in different control approaches aiming to improve the tracking capability of the rotational stage. It shows that the \abbrIRC method could be used to efficiently control the rotational stage. Moreover it shows that a harmonic cancellation method could be used to increase the tracking accuracy by canceling known harmonic disturbances. The harmonic cancellation method (the \abbrRFDC) was implemented in this thesis and proposed as an add-on to the present control algorithm.
