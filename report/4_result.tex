% !TEX root = main.tex
%preliminära resultat som kan demonstreras vid halvtidskontroll
\chapter{Result}\label{cha:result}
This section describes the results considering performance, robustness and stability with respect to the different control approaches. The performance of each approach will first be presented individually and in a comparison in the end.


\section{Simulation Results}
\subsection{Model Reference Adaptive Control}




\section{Experimental Results}
\subsection{Setup}
The experiments have been conducted on the considered rotational stage described in Section~\ref{sec:rotational_stage}. A National Instruments PXI communicates with the Collimator and the rotaional stage, responsible for acquisition and control. To drive the rotaional stage it outputs a voltage between [-1, 7.5] V, this is amplified by a linear amplifier with a gain of \unit{20}{\volt/\volt} resulting in a [-20, 150] V signal input on the rotaional stage. The yaw angle, measured by the interferometric system is converted and sent to the PXI. The control loop is running at \unit{2}{\kilo\hertz}.
