% !TEX root = main.tex
\chapter{Introduction}\label{cha:intro}

\section{Background}
High precision positioning systems are vital in e.g. scanning tunneling microscopes (\abbrSTM), atomic force microscopes (\abbrAFM) and in semiconductor lithography. In AFM, for instance, high precision positioning is needed to control the position of a scanning probe with atomic-scale resolution. In the UA9 Experiment located at \abbrCERN (European Organization for Nuclear Research) a high precision positioning system is required for the control of a piezo-actuated rotational stage.

Smart materials such as piezoelectric and magnetostrictive materials are commonly used in precision actuators today due to their ability to convert electrical energy into mechanical energy. %%where the latter material changes dimensions during the process of magnetization from an external field typically generated by a current passing through a coil. %%
In cases where a relatively small displacement range is required (travel ranges up to \unit{500}{\micro\meter}) a piezo electric device is the actuator of choice due to its fast response, high resolution and its ability to generate large mechanical forces for small amounts of power in compact designs \citep{SurveyOfControlIssues:2007}. The rotational stage at CERN uses a piezo electric linear stack actuator to displace a flexible lever arm mechanism which generates the rotational movement. The piezo-actuator stage together with its amplifying structure gives the device an rotational range of \unit{20}{\milli\rad}.

\section{Purpose and Goal}
Crystalline solids have the ability to constrain the directions that particles take as they pass through, this is commonly called the "channelling" property. The UA9 collaboration at \abbrCERN is investigating how tiny bent crystals can help to steer particle beams in modern hadron colliders such as the Large Hadron Collider (\abbrLHC) \citep{WebsiteUA9:2016}. In circular colliders, such as the \abbrLHC, particles tends to drift outwards creating a beam halo that may cause harm to the system. By using bent crystals, halo particles can be efficiently extracted from the beam and collected by absorbers.
One major difficulty that aries is that the higher the energy of the particle, the lower the angular acceptance for channeling. Hence, a high precision positioning mechanism with a high angular accuracy is required. The rotational stage at \abbrCERN is of necessity to be able to track reference trajectories at ramp rates of \unit{100}{\micro\radianpersecond} and reject external disturbances to maintain a maximum tracking error of $\pm$\unit{1}{\micro\rad}.

This project aims to identify the possible control approaches that could be applicable to this problem to achieve the desired performance.

\section{Related work}

\section{Outline}

en preliminär problemformulering satt i relation till litteraturbasen
