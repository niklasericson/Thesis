% !TEX root = main.tex
\chapter{Introduction}\label{cha:intro}

\section{Background}
The piezoelectric effect is a phenomenon that arises in certain solid materials when an electric potential is generated in response to applied mechanical stress. The effect was first discovered by Jacques and Pierre Curie in 1880 when they found that applying pressure to a quarz crystal results in an electrical charge. Today, the effect is commonly encountered in daily life in forms such as lighters, buzzers and loudspeakers.  \citep{SurveyOfControlIssues:2007} \citep{Piezo:2008}.

Smart materials such as piezoelectric and magnetostrictive materials are commonly used in precision actuators today due to their ability to convert electrical energy into mechanical energy. In cases where a relatively small displacement range is required (travel ranges up to \unit{500}{\micro\meter}) a piezo electric device is the actuator of choice due to its fast response, high resolution and its ability to generate large mechanical forces for small amounts of power in compact designs \citep{SurveyOfControlIssues:2007}.

High precision positioning systems are vital in e.g. scanning tunneling microscopes (\abbrSTM), atomic force microscopes (\abbrAFM) and in semiconductor lithography. In \abbrAFM, for instance, high precision positioning is required to control the vertical position of the scanning probe to keep the force constant between the sample surface and the probe tip. An topographical image of the sample is obtained by raster-scanning the probe over the sample surface and plotting the vertical displacement against the probe's x-y position. A positioning system that keeps the force constant down to an atomic-scale resolution is thus inevitable in order to obtain a high resolution image without damaging the sample \cite{SurveyOfControlIssues:2007}.

In the UA9 Experiment located at \abbrCERN (European Organization for Nuclear Research) a high precision positioning system is required for the control of a piezo-actuated rotational stage. The stage uses a piezo electric linear stack actuator to displace a flexible lever arm mechanism which generates the rotational movement. The piezo-actuator stage together with its amplifying structure gives the device an rotational range of \unit{20}{\milli\rad}.

\section{Purpose and Goal}
Crystalline solids have the ability to constrain the directions that particles take as they pass through, this is commonly called the "channelling" property. The UA9 collaboration at \abbrCERN is investigating how tiny bent crystals can help to steer particle beams in modern hadron colliders such as the Large Hadron Collider (\abbrLHC) \citep{WebsiteUA9:2016}. In high energy colliders, such as the \abbrLHC, particles drift outwards creating a beam halo. These particles surrounding the beam, can be lost and cause damage to sensitive parts in the accelerator. By using bent crystals, halo particles can be efficiently extracted from the beam and collected by absorbers further away, reducing the complexity of the system. One major difficulty that aries is that the higher the energy of the particle, the lower the angular acceptance for channeling. Hence, a high precision positioning mechanism with a high angular accuracy is required. The rotational stage at \abbrCERN is of necessity to be able to track reference trajectories at ramp rates of \unit{100}{\micro\radianpersecond} and reject external disturbances to maintain a maximum tracking error of $\pm$\unit{1}{\micro\rad}.

This project aims to identify the possible control approaches that could be applicable to this problem to achieve the desired performance.

\section{Related work}

\section{Outline}

en preliminär problemformulering satt i relation till litteraturbasen
