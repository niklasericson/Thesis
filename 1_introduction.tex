% !TEX root = main.tex
\chapter{Introduction}\label{cha:intro}

High precision positioning systems are vital in e.g. scanning tunneling microscopes (STM), atomic force microscopes (AFM) and in semiconductor lithography. In AFM, for instance, high precision positioning is needed to control the position of a scanning probe with atomic-scale resolution. In the UA9 Crystal Collimation Experiment located at CERN (European Organization for Nuclear Research) a high precision positioning system is required for the control of a piezo-actuated rotational stage.

Smart materials such as piezoelectric and magnetostrictive materials are well used in precision actuators today due to their ability to convert electrical energy into mechanical energy. %%where the latter material changes dimensions during the process of magnetization from an external field typically generated by a current passing through a coil. %%
In CERN's case where a relatively small displacement range is required (piezo electric stack actuators with travel ranges up to 500 $\mu m$ are available) a piezo electric device is the actuator of choice due to its fast response, high resolution and its ability to generate large mechanical forces for small amounts of power in compact designs \citep{SurveyOfControlIssues:2007}. Hence, CERN's rotational stage uses a piezo electric linear stack actuator to displace a flexible lever arm mechanism generating the rotational movement.

requires rotational positioning with a high accuracy

The device, with a rotational range of 20 mradians, should be able to track reference trajectories at ramp rates of 100 urad/s and reject external disturbances with a maximum tracking error of +/-1 urad.



en preliminär problemformulering satt i relation till litteraturbasen
