\documentclass[
  utf8,%     More capable input encoding than latin-1.
  % parskip,%  For vertical whitespace between paragraphs.  This comes down to more than just using parskip.sty, so it's better to use this class option.
  % S5MP % If you intend to really use margin paragraphs (not recommended!).
%  crop,%     Produce output with crop marks and paper size A4.  Liu-Tryck should like this.  Automatically adds information, including the physical page number, at the top of each page.
       %     Add option 'noInfo' to suppress the info at the top of each page when using option 'crop'.
  % Font options: 'kp' (default), 'times', 'lm'.  The KpFonts (loaded using 'kp'), is the most complete font among the provided options.  Among other, it supports slanted small caps.  See rtthesis.cls for more details regarding the font options.
  largesmallcaps,intlimits,widermath,% Good options to KpFonts.
  sharecounter,nobreak,definition=marks,%  See comments in the results chapter of this document for more information on these options!
  %numbers, % If you want to cite references by numbers, use this option.
  noparts% Use option 'noparts' if you do not make use of part divisions.
]{rtthesis}

\usepackage{mythesis}

% !TEX program = pdflatex
% !TEX root = main.tex


\begin{document}
\selectlanguage{english}
\makeFrontPage
\frontmatter
\maketitle
\makeLibraryPage{Det här som vi har hållit på med är jätteviktigt faktiskt och det vi gjort blev bara sååå bra.  Kanske inte helt otippat, men det glass är sååå gott!

Förresten har vi blivit bäst på att skriva rapporter, så nu ska ska vi inte gå in närmare på några detaljer såhär i sammanfattningen.
}

\begin{abstract}[swedish]
  Det här som vi har hållit på med är jätteviktigt faktiskt och det vi gjort blev bara sååå bra.  Kanske inte helt otippat, men det glass är sååå gott!

Förresten har vi blivit bäst på att skriva rapporter, så nu ska ska vi inte gå in närmare på några detaljer såhär i sammanfattningen.

\end{abstract}
\begin{abstract}[english]
  The Equipment Controls and Electronics section at \abbrCERN is developing a high precision piezo-actuated rotational stage for the UA9 crystal collimation project. Several control-related issues arising with the complexity and operational environment of the system make it difficult to design a controller that achieves the desired performance. This thesis aims to identify different control approaches that can be applicable to this problem.

%If your thesis is written in English, the primary abstract would go here while the Swedish abstract would be optional.

\end{abstract}
\begin{acknowledgments}
  First of all, I would like to thank \abbrCERN and the \abbrENSTIECE section for giving me this opportunity. 

  \addvspace{1em}
  \begin{flushright}
    \textit{%
      Linköping, Augusti 2016\\
      Niklas Ericson%
    }
  \end{flushright}
\end{acknowledgments}

\tableofcontents
\begin{notation}% Passing the option "old" to the notation environment will redefine the notationtabular environment so that it produces an old style LaTeX tabular instead of a ctable.sty style tabular.
  \centering

  % \begin{notationtabular}{Några mängder}{Notation}{Betydelse}
  %   $\naturals$ & Mängden av naturliga tal \\
  %   $\reals$ & Mängden av reella tal \\
  %   $\complexes$ & Mängden av komplexa tal \\
  % \end{notationtabular}

  \begin{notationtabular}{Abbreviations}{Abbreviation}{Meaning}
    \abbrCERN\index{CERN@\abbrCERN!abbreviation} & European Organization for Nuclear Research \\
    \abbrSTM\index{STM@\abbrSTM!abbreviation} & Scanning Tunneling Microscope \\
    \abbrAFM\index{AFM@\abbrAFM!abbreviation} & Atomic Force Microscope \\
    \abbrLHC\index{LHC@\abbrLHC!abbreviation} & Large Hadron Collider \\
    \abbrPEA\index{PEA@\abbrPEA!abbreviation} & Piezoelectric Actuator \\
    \abbrPID\index{PID@\abbrPID!abbreviation} & Proportional, Integral, Derivative (controller) \\
    \abbrDOF\index{DOF@\abbrDOF!abbreviation} & Degrees of Freedom \\
    \abbrPRBS\index{PRBS@\abbrPRBS!abbreviation} & Pseudo Random Binary Sequence \\
    \abbrTF\index{TF@\abbrTF!abbreviation} & Transfer Function \\
    \abbrQFT\index{QFT@\abbrQFT!abbreviation} & Quantitative Feedback Theorem \\
  \end{notationtabular}
\end{notation}


\mainmatter
\chapter{Introduktion}\label{cha:intro}

en preliminär problemformulering satt i relation till litteraturbasen

\chapter{Method}\label{cha:metod}

en preliminär beskrivning av angreppssätt

planerad litteraturbas

en tidplan för examensarbetets genomförande inklusive planerade datum för halvtidskontroll och framläggning

\chapter{Resultat}\label{cha:metod}

preliminära resultat som kan demonstreras vid halvtidskontroll

\include{conclusion}

\part*{Appendix}
\appendix
\chapter{definitions}\label{cha:definition}


\backmatter

\bibliography{IEEEfull,myrefs}

\printindex

\end{document}
